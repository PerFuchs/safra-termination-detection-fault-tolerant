\section{Introduction}
This technical report presents an experiment using our fault-tolerant version of Safra's termination detection algorithm for rings (short SafraFT)~\cite{safraFT2018}
as control algorithm to detect termination of two different basic algorithms: a fault-tolerant version of Chandy-Misra's routing algorithm and Afek-Kutten-Yung's self-stabilizing spanning tree algorithm.
Both basic algorithms build a tree containing all nodes in a distributed system.
However, they cannot determine termination by themselves; that is, they cannot detect if they succeeded to build tree themselves, but they need a termination detection algorithm to verify that they completed their goal.
This is the task of SafraFT; when it detects termination, it announces this finding to all nodes of the distributed system.
SafraFT is a version of Safra's termination detection algorithm~\cite{safraFS} that can deal with fail-stop failures\footnote{Fail-stop is a failure model that models failures as a binary, permanent property: a node is either running as normal or has failed; if a node fails, it does not send or react to any messages anymore and cannot be repaired.}. 
That means it detects termination correctly even in the presence of fail-stop failures.

The following report is not self-contained.
I recommend to read these texts first:
\begin{itemize}
	\item the chapters about Chandy-Misra, Afek-Kutten-Yung, Safra's termination detection algorithm and a definition of termination detection in \textit{Distributed Algorithms an Intuitive Approach} by Fokkink~\cite{fokkink:2018}
	\item the technical report \textit{Fault-Tolerant Termination Detection with Safra’s Algorithm} by Fokkink and Karlos~\cite{safraFT2018} for a detailed explanation of SafraFT.
\end{itemize}
The next paragraphs give an overview on the structure of this report and the experiments conducted.

The fault-tolerant Chandy-Misra version is developed for this project as an extension of the fault sensitive Chandy-Misra algorithm described in~\cite{fokkink:2018} on page 57.
An explanation of the extension and further important implementation decisions can be found in \cref{sec:methods}.

The experiment aims to
\begin{itemize}
	\item back up our claim that SafraFT is correct by using it in a realistic setting and verifying that termination is detected in a timely manner after actual termination occurred
	\item compare the performance of SafraFT to the fault-sensitive Safra implementation (abbreviated SafraFS) from \cite{demirbas2000optimal}
	\item demonstrate the ability of SafraFT to handle networks of 25 up to 2000 nodes in a fault-free, 1 to 5 faults, and highly faulty (90\% node failure) environment
\end{itemize}

Towards this aim, I measure the following dependent variables
\begin{itemize}
	\item total number of tokens sent
	\item number of tokens sent after the basic algorithm terminated
	\item number of backup tokens sent
	\item average token size in bytes
	\item processing time for Safra's procedures
	\item time spent processing the basic algorithm's procedures
	\item wall time for the complete computation
	\item wall time  between termination of the basic algorithm and detection of the fact
\end{itemize}

All these metrics are measured within the following environments:
\begin{itemize}
	\item network size of 25, 250, 500, 1000 and 2000 nodes
	\item using SafraFS and SafraFT
	\item in a fault-free environment
	\item for SafraFT additionally with 1 - 5 and up 90\% node failures (simulating nearly fault-free networks and highly faulty environment)
\end{itemize}

I do not aim to show the exact relationships between the dependent variables and the independent variables, e.g. the relationship between backup tokens send and the number of faults.
This is because the exact relationship depends heavily on the basic algorithm, the network and even the hardware the system is running on.
Therefore, detailing the dependence would not be helpful to anyone considering to apply our algorithm to his system.
However, this experiment should enable the reader to judge if SafraFT could be used for his system and convince him that SafraFT performance is comparable to that of SafraFS in a fault-free environment (except for its higher bit complexity).
Furthermore, this report aims to show how SafraFT behaves in a faulty environment.

During my experiments, I found two implementation issues in SafraFT.
We could address those without much ado and the results presented in~\cref{sec:results} are runs on the fixed version of SafraFT.
The implementation issues and the fixes are explained in \cref{sec:safraBugs}.

Before performing this experiment, George Karlos~\cite{karlos} applied SafraFT to an emulated basic algorithm in a multi-threaded environment.
These experiments showed evidence towards the correctness and a good performance of SafraFT.

The experiments presented here were performed on the recommendation of a reviewer of~\cite{safraFT2018} to add a `compelling end-to-end application'.
