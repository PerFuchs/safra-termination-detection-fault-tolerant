\section{Introduction}
This technical report presents an experiment using our fault-tolerant Safra version (short SafraFT)~\cite{safraFT2018}
to detect termination of a fault tolerant version of Chandy Misra's routing algorithm and Afek-Kutten-Yung's self-stabilizing spanning tree algorithm.
The fault-tolerant Chandy Misra version is developed for this project as an extension of the fault sensitive Chandy Misra algorithm described in~\cite{fokkink:2018} on page 57.
An explanation of the extension and further important implementation decisions can be found in \cref{sec:methods}.

The experiment aims to
\begin{itemize}
	\item backup our claim that SafraFT is correct by using it in a realistic setting and verifying that termination is detected in a timely manner after actual termination occurred
	\item compare the performance of SafraFT to the fault sensitive Safra implementation (abbreviated SafraFS) from \cite{demirbas2000optimal}
	\item demonstrate the ability of SafraFT to handle networks of 25 and up to 2000 nodes in a fault-free, 1 to 5 faults and highly faulty (90\% node failure) environment
\end{itemize}

Towards this aim, I measure the following dependent variables
\begin{itemize}
	\item total number of tokens send
	\item number of tokens send after the basic algorithm terminated
	\item number of backup tokens send
	\item average token size in bytes
	\item processing time for Safra's procedures
	\item time spent processing the basic algorithm's procedures
	\item wall time for the complete computation
	\item wall time  between termination of the basic algorithm and detection of the fact
\end{itemize}

All these metrics are measured within the following environments:
\begin{itemize}
	\item network size of 25, 250, 500, 1000 and 2000 nodes
	\item using SafraFS and SafraFT
	\item in a fault-free environment
	\item for SafraFT additionally with 1 - 5 and up 90\% node failures (simulating nearly fault-free networks and highly faulty environment)
\end{itemize}

I do not aim to show the exact relationships between the dependent variables and the independent variables e.g. the relationship between backup tokens send and the number of faults.
This is because the exact relationship depends heavily on the basic algorithm, the network and even the hardware the system is running on.
Therefore, detailing the dependence would not be helpful to anyone considering to apply our algorithm to his system.
However, this experiment should enable the reader to judge if SafraFT could be used for his system and convince him that SafraFT performance is comparable to that of SafraFS in a fault-free environment (except for its higher bit complexity).
Furthermore, this report aims to show how SafraFT behaves in a faulty environment.

During my experiments, I found two bugs in SafraFT.
We could fix those without much change and the results presented in~\cref{sec:results} are runs on the fixed version of SafraFT.
The bugs and the fixes are explained in \cref{sec:safraBugs}.

Before performing this experiment George Karlos applied SafraFT to a simulated basic algorithm in a multi-threaded environment emulating a distributed system.
These experiments showed strong evidence towards correctness and scalability of SafraFT.
The implementation, technical report and results can be found in George's bachelor thesis~\cite{karlos}.

The experiment presented here is performed on the recommendation of a reviewer to add `compelling end-to-end application'.
