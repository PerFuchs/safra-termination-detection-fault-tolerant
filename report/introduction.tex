\section{Introduction} 
This technical report presents an experiment using our fault tolerant Safra version (short SafraFT) \cite{ourpaper}
to detect termination of a fault tolerant Chandy Misra routing algorithm.
The fault-tolerant Chandy Misra version is developed for this project as an extension of the fault sensitive Chandy Misra algorithm described in \cite{Fokkink:2018} on page 57.
An explanation of the extension and further important implementation decisions can be found in \cref{sec:methods}.

The experiment aims to
\begin{itemize}
	\item backup our claim that SafraFT is correct by using it in a realistic setting and verifying that termination is detected in a timely manner after actual termination occurred
	\item compare the performance of SafraFT to a fault sensitive Safra implementation (abbreviated SafraFS) from \cite{demirbas2000optimal}
	\item demonstrate the ability of SafraFT to handle networks of 25 and up to 2000 nodes in a fault-free, 1 to 5 faults and highly faulty (90\% node failure) environment
\end{itemize}

Towards this aim, I measure the following dependent variables
\begin{itemize}
	\item total number of tokens send
	\item number of tokens send after the basic algorithm terminated
	\item number of backup tokens send
	\item average token size in bytes
	\item processing time for Safra's procedures
	\item time spent processing the basic algorithm's procedures 
	\item wall time for the complete computation
	\item wall time  between termination of the basic algorithm and detection of the fact
\end{itemize} 

All these metrics are measured within the following environments:
\begin{itemize}
	\item network size of 25, 250, 500, 1000 and 2000 nodes
	\item using SafraFS and SafraFT
	\item in a fault-free environment
	\item for SafraFT additionally with 1 - 5 and up 90\% node failures (simulating nearly fault-free networks and highly faulty environment)
\end{itemize}

I do not aim to show the exact relationships between the dependent variables and the independent variables e.g. the relationship between backup tokens send and the number of faults.
This is because the exact relationship depends heavily on the basic algorithm, the network and even the hardware the system is running on.
Therefore, detailing the dependence would not be helpful to anyone considering to apply our algorithm to his system.
However, this experiment should enable the reader to judge if SafraFT could be used for his system and convince him that SafraFT performance is comparable to that of SafraFS in a fault-free environment (except for its higher bit complexity).
Furthermore, this report aims to show how SafraFT behaves in a faulty environment.

\begin{comment}
Hypothesis
Safra compares to safra without signficant changes in with no failures on all network sizes
tokens
tokens after termination

no signifcant change in token forwarding
time ??
no hard, complicated computations
time after termination
no hard, complicated computations and no change in token forwarding

our Safra has higher bit complexity 
increases linearly with network size network size of 1 == then upwards linearly

%TODO    
Influence from faults
backup tokens == number of faults - not correct
difference caused by backup tokens / network size under same fault conditions?
reason:  // Not to write here
node gets to know about failing node because of token
e.g. if its a bit further out and it gets the crashed message later
node crash behind each other in the ring and only one backup token is send?
Token count increases for a low level of faults as it colors nodes black and incurs in further rounds when node detects crash or when nodes gets to know of crash of successor    x
These are mostly done during computation because CM is not processing heavy and gets passive a lot
-> No change on tokens after termination
Token count and time decrease for a high level of fault as network size is reduced. 
Same for time spent

Average token size grows as the crashes are propagated by the token each of them takes one round so impact on average token size depends on network size and the signficance also on number of rounds
more noticable for many faults but than there are also less tokens also in that round
\end{comment}



Before performing this experiment George applied SafraFT to a simulated basic algorithm in a multi-threaded environment emulating a distributed system.
These experiments showed strong evidence towards correctness and scalability of SafraFT.
The implementation, technical report and results can be found here: \cite{georgework}.

The experiment presented here is performed on a reviewers recommendation to add `compelling end-to-end application'.
