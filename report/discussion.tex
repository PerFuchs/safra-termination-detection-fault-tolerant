\section{Discussion}
First, this section concludes what the results imply about SafraFT. After, I discuss limitations and improvements of my approach.

The most important aim of the experiment is to show correctnes of SafraFT.
Towards this aim, the results leave no doubt that SafraFT behaved correctly in all 1500
%number 
runs.
These runs cover a wide range of different environments from small networks of 50 nodes to big networks with 2000 instances and for each network size the fault-free case, as well as, two quite different fault scenarios.
The basic algorithm is carefully chosen to exhibit an interesting and varied message pattern in terms of the number of messages and communication partners.
Furthermore, Chandy Misra leads to many changes between the active and the passive state of each node.
The events on which to simulate node failure are deliberately chosen to put high pressure on SafraFT ability to recover faults and should trigger many different configurations.
All in all, the situations tested are designed to test SafraFT thoroughly under stress.
Still, termination is detected in every case with no case of early detection.

However, the experiment shows that the definition of termination used to develop SafraFT, although very common and widely accepted, does not cover all situations arising in praxis.
The definition assumes passive nodes only become active when they receive a basic message.
In reality, nodes can also change to an active state when they detect the crash of another node and need to take corrective actions.

When I realized this, I extended the definition of termination as described in \label{extended-definition} and evaluated the results of the experiment according to that extension.
This second evaluation showed that this does not happen often (12 out of 1500 runs) and only in 9 of these cases it leads to a corrupted result of the basic algorithm.
%number

Such cases are unlikely in praxis because it is less likely there than in our experiment that a fault happens just before termination because in our experiment many faults are triggered when a token is forwarded or received while in praxis most fault ought to happen within the basic algorithm or idle time because the vast majority of the time is spent for these purposes.

Furthermore, one can actively reduce the chance of early termination detection by using a failure detector that propagates failures fast to all nodes and then enforces an extra token round.    

% TODO can we write something here about other being not better? And that this is an inherent problem of termination under the presence of failures?
The practical comparison between SafraFT and SafraFS shows that there is no change in message complexity and confirmed the hypothesis that bit complexity raises linearly with the network size. 
This higher bit complexity is the reason why SafraFT takes longer to detect termination after actual termination and leads to a higher processing time overhead over the basic algorithm than for SafraFS.

I would like to point out that my experimental setup imposes that the time taken by SafraFT to detect termination is unreasonably long.
This is because Chandy Misra completes its task in no time and then SafraFT starts the seemingly long process of completing a token round.
If I had chosen a basic algorithm that takes multiple hours to complete, the seconds taken by SafraFT to detect termination would be seen in a different perspective.
Furthermore, the low processing time overhead (less than 5\%) of SafraFT plays a bigger role in long-running jobs.
%number

Towards SafraFT performance in the presence of faults, I conclude that the number of backup tokens issued is not limited by the number of faults but the average is lower than the number of faults.
Also, the increase in bytes per token is only constant (8 bytes) for a few faults and grows only linearly with the network size for 90\% node failure.
It is difficult to predict general relationships between the number of faults, network size and other measured metrics e.g. time or token sent.  
Still, the experiments clearly show that even in big networks and with 90\% node failure the algorithm performs well and none of the metrics show exponential growth.
% TODO include overhead over basic algorithm for faulty runs

In total, the experiment reaches its goals, it presented strong evidence towards correctness of SafraFT, gives a realistic comparison between the fault tolerant version and the original version of Safra and demonstrates SafraFT's ability to handle faults.

In the next paragraphs, I point out possible improvements for this project.

We would have liked to run repetitions with even bigger networks than 2000 nodes.
This was hindered by the limited physical resources at hand - only 42 nodes with 8 cores each.
Therefore, even for 2000 instance multiple of these had to run in parallel on the same core. 
That lead to slow runs for big networks and serious scalability issues with IPL (the communication library used).
Furthermore, IPL does not provide rather important primitives as broadcasting or barriers.
So both had to be implemented by me during the project.
Especially, implementing robust and performant barriers showed to be more work than expected as one IPL feature (signals) did break with 2000 nodes and it demonstrated to be impossible to connect each node to one master node to orchestrate actions.
Hence, I recommend using more machines and a more robust communication library if one wants to use my implementation for bigger networks.

Although showing the general relationship between network size and behaviour in the presence of faults is not the main goal of this project, I hoped to come to more precise conclusions regarding this matter.
To do so one would need to run the experiments on bigger networks and preferably concentrate on a specific but relevant fault scenario.
Even with this additional data, it could prove very hard to attain general conclusions as faults do not only influence SafraFS but the whole system e.g. a fault always leads to fewer nodes in the system and therefore changing the network size or it leads to different behaviour of the basic algorithm which in term changes Safra's activity.
% TODO: Comparision to George experiments?